Modellbildung der Strecke durch mathematische Beschreibungen der Wirkungszusammenh�nge zwischen den Systemgr��en, die f�r die Aufgabenstellung relevant sind.

Ein Modell ist eine aufgabenspezifische Vereinfachung der Realit�t. In der RT bew�hrte Modellierungsform:

\section{Darstellung der Strecke als Strukturbild (Blockschaltbild)}
\subsection{Beispiel: Permanten erregter Gleichstrommotor}
\begin{itemize}
	\item Ger�teschema: (siehe Beiblatt 4)
	\item Systemdarstellung: 
	\begin{center}
	\begin{tikzpicture}[node distance=2.5cm,auto,>=latex']
		\node [system, pin={[arrowthick]above:$U_L$}] 	(system) 						{Motor};
		\node 							(in) 	[left of=system,  node distance=2cm, coordinate]{system};
		\node [coordinate] 					(out) 	[right of=system, node distance=2cm]		{};
	
		\path[->,thick] (in) edge node {$U_A$} (system);
		\draw[->,thick] (system) edge node {$\omega$} (out);
	\end{tikzpicture}
	\end{center}
\end{itemize}
