\begin{thms}
	Modellbildung der Strecke durch mathematische Beschreibungen der Wirkungszusammenh�nge zwischen den Systemgr��en, die f�r die Aufgabenstellung relevant sind.
\end{thms}

Ein Modell ist eine aufgabenspezifische Vereinfachung der Realit�t. In der RT bew�hrte Modellierungsform:

\section{Darstellung der Strecke als Strukturbild (Blockschaltbild)}
\subsection{Beispiel: Permanten erregter Gleichstrommotor}
\begin{itemize}
	\item Ger�teschema: (siehe Beiblatt 4)
	\item Systemdarstellung:
		\begin{center}
			\includegraphics[width=200px]{graphics/gleichstrommotor.pdf}
		\end{center}
	\item Ermittlung der beschreibenden Gleichungen
		\subitem Ankerstromkreis
		\begin{align}
		u_L=L\frac{di_A}{dt} \rightarrow \frac{di_A(t)}{dt}=\frac{1}{L}u_L(t) \stackrel{\int_0^t}{\rightarrow}i_A(t)=i_A(0)+\frac{1}{L_A}\int_0^t u_L(\tau)d\tau \\
		u_A=u_R+u_L+u_{ind} \rightarrow u_L(t)=u_A(t)-u_R(t)-u_{ind}(t) \\
		u_R(t)=R_Ai_A(t) \\
		U_{ind}=c\phi\omega(t)
		\end{align}
		\subitem Rotierender Anker und Welle
		\begin{align}
		J\dot{\omega}=M_{\sum} \rightarrow \dot{\omega}(t)=\frac{1}{J}M_{\sum}(t)\stackrel{\int_0^t}{\rightarrow}\omega(t)=\omega(0)+\frac{1}{J}\int_0^t M_{\sum}(\tau)d\tau \\
		M_{\sum}(t)=M_A(t)-M_L(t) \\
		M_A(t) = c \phi_F L_A(t)
		\end{align}
	\item �bersetzung der Gleichungen ins Strukturbild {TODO:BILD}
\end{itemize}