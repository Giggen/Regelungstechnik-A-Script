am Beispiel des Stellverhaltens mit der �F $G(s) = G_1 G_2 $

\section{Betrachtung der Strecken �F}
\subsection{Polynomform der �F}
  \[ G(s) = \frac{b_ms^m + b_{m-1}s^{m-1} + ... + b_0}{a_ns^n + a_{n-1}s^{n-1} + ... + a_0} = \frac{Z(s)}{N(s)} \], wobei
 \begin{itemize}
  \item  $ a_i, b_j \in \mathbb{R} \text{ mit } a_n \not= 0 \wedge b_m \not= 0$ 
  \item m = grad $Z(s)$: Z�hlergrad
  \item n = grad $N(s)$: Nennergrad bzw. Systemordnung
  \item $\delta = m-n$: Differenzgrad bzw. Relativgrad des Systems
  \end{itemize}
  \[  	\left.\begin{matrix}
	    \delta > 0 \text{ verz�gerndes} \\
	    \delta = 0 \text{ nicht verz�gerndes} \\
	    \delta < 0 \text{ differenzierendes} \\
	\end{matrix}\right\}
	\text{Verhalten des Systems}
  \]
 
Da im allgemeinen $n \geq m$, d.h. $\delta \geq$  ist $\delta$ ein Ma� f�r die verz�gernde Wirkung eines Systems 
(siehe \bb{13})

\subsection{Pol- Nullstellenform der �F}
Sind $n_1,...,n_m$ bzw. $p_i,...,p_n$ die (reellen oder konjugiert komplexen) Wurzeln von $Z(s)=0$ bzw. $N(s)=0$ so folgt:
 \[ G(s) = K \frac{\prod_{j=1}^m (s-n_j)}{\prod_{i=1}^n} \text{mit} K= \frac{b_m}{a_n} \]
 Dabei gilt: \begin{itemize}
              \item $ \left| G(s=n_j) \right| = 0, \text{ falls} n_j \not= p_1,...,p_n $ \\
		    Dieses $n_j$ ist eine NS $s_0$ der �F
	      \item $ \left| G(s=p_i) \right| = \infty, \text{ falls} p_i \not= n_1,...,n_m $ \\
		    Dieses $p_i$ ist eine PS $s_\infty$ der �F
             \end{itemize}
Im weiteren gelte stets: $p_i \not= n_j \forall i,j$ ! \\
\begin{itemize}
 \item entspricht dem gemeinsamen Aufbau beider Beispielsysteme, d.h \\
	\TODO{Strukturbild}
  \item obige Form der �F erm�glict ihre graphische Darstellung in der komplexen s-Ebene als PN-Diagramm. 
	\TODO{Bild}
  \item Zusammenh�nge zwischen Pol- und Nullstellenlage und Zeitverhalten des Systems: 
  \subitem $ \operatorname{Re} \lbrace p_i \rbrace < 0 \forall_i \leftrightarrow \text{System mit} \delta \geq 0 
	\text{BIBO-stabil} $ 
  \item Weiterhin Betrachtung der Sprungantwort (bei Annahme endlicher Pole $p_i \not 0 $
	\[ y_\sigma(s) = G(s) \frac{1}{s} \stackrel{PBZ}{=} \frac{r_0}{s} + \sum_{k=1}^N \frac{r_i}{s-p_i} \]
	\[ \Laplace y_\sigma(t) = (r_0 + \sum_{k=1}^N r_i e^{p_it}) \sigma (t)	\]
  \item Zu den Polstellen geh�rende Zeitverl�ufe: siehe \bb{14}
  \item $\rightarrow$ Beitr�ge stabiler Pole klingen ab, und zwar umso schneller, je weiter links sie liegen.
  \subitem Weit links liegende Pole sind nicht dominant (d.h. nicht pr�gend f�r das Zeitverhalten und somit ggf. vernachl�ssigbar),
	   wohl aber nahe der j-Achse liegende sowie instabile Pole (sofern nicht Nullstellen nahe dabei, was
	  \[ r_\nu = \left. y_\sigma(s)(s-p_\nu) \right|_{s=p_\nu} \stackrel{\nu>0}{=} K\frac{\prod_{j} (p_\nu - n_j)}{p_r \prod_{j} (p_\nu - p_i)} \approx 0 \]
	   zur Folge hat, falls $n_j \approx p_\nu$ gilt.
  \item Weiterhin gilt: $r_0 = \sigma(s=0) = K \frac{\prod_j (-n_j)}{\prod_i (-p_i)}$ falls ein $n_j =0$, dann $r_0 = 0$, d.h $y(t) \rightarrow 0 \forall u(t) = N_0 \sigma(t)$ 
  mit $U_0$ beliebig. 
 \item Hat ein System eine NS bei $s=0$, so sind y-Station�werte $\not= 0$ nicht erzielbar!
 \item Hat eine stabile Strecke K NS mit $\operatorname{Re} \lbrace n_j \rbrace > 0 $ (nichtminimalphasige Strecke), so
      so schneidet $y_\sigma(t)$ f�r $t>0$ k-mal die Nulllinie. 
 \item TODO{Strukturbild}
 \item gezielte Beeinflussung schwieriger als im minimalphasigen Fall
\end{itemize}

\subsection{Zeitkonstantenform der �F}
\underline{Folgt durch ausklammern der Pole und NS $\not= 0$}
\begin{itemize}
 \item reeller Pol $pi \not= 0$: $s-p_i = -p_i(1+ \frac{1}{-p_i}s)$ mit $T_i = \frac{1}{-p_i}( >0 \text{ f�r } p_i < 0)$:\\
      Zeitkonstante zum reellen Pol $p_i$
 \item kleines gro�es $T_i >0 \rightarrow $ kleine gro�e Abklingzeit (siehe \bb{14} )
 \item konjugiert komplexes Polpaar $p_i; p_n = p_k^*$:\\
 \begin{align}
      (s-p_i)(s-p_i^*) &= p_i p_i*(1+ \frac{1}{-p_i}s)(1+\frac{1}{-p_i^*}s) \\
		       &= p_i p_i^*(1+ \frac{-(p_i + p_i^*)}{p_i p_i^*}s + \frac{1}{p_i p_i^*}s^2) \\
		       &= \left| p_i \right|^2 (1 + \frac{-2\sigma_i}{\left| p_i \right|^2} s + \frac{1}{\left| p_i \right|^2}s^2) \\
		       &= \frac{1}{T_i^2}(1 + 2D_iT_is + T_i^2s^2)
\end{align}
\[  	\text{mit } 
	\left\{ \begin{matrix}
	    T_i > 0 = \frac{1}{\left| p_i \right|^2} \text{ = Zeitkonstante} \\
	    D_i = \frac{-\sigma_i}{\left| p_i \right|^2}, \left| D_i \right|^2 \text{ = D�mpfung}\\
	\end{matrix}\right. 
  \]
\item kleines/ gro�es $\frac{T_i}{D_i} = \frac{1}{\sigma_i} > 0 \bb{14} \to $ kleine/ gro�e Abklingzeit der Einh�llenden 
\item gro�es/ kleines $D_i \to $ geringe/ hohe Schwingneigung (siehe \bb{7})
\item entsprechend $T_{zj} = \frac{1}{-n_j} \text{ bzw. } T_{zj} = \frac{1}{\left| n_j \right|}, D_{zj} = \frac{-\operatorname{Re}\lbrace n_j \rbrace}{\left| n_j \right|}$ \\
      bei reellen bzw. konjugiert komplexen Nullstellen
\item Damit resultiert als �F nach dem Ausklammern der Pole und Nullstellen $\not= 0$: 
\[G(s) = K \frac{(1-T_{zj}s)...(1+D_{z\mu}T_{z\mu}s + T_{z\mu}^2s^2)}
	  {s^q (1+T_1s)(1+t_2s)...(1 + 2D_\nu T_{\nu s} s + T_\nu^2 s^2)}
\]
mit $K = k \frac{\prod_n (n_j)}{\prod_{p_i \not= 0} (-p_i)}$ : Verst�rkungsfaktor der Strecke (meist Annahme: $K>0$)
\item q: Anzahl der freien Integrierer/ Differenzierer (bei q<0) in der Strecke (Meist $q=0,1,2$ (Strecke mti P-/I-/ Doppel-I Verhalten))
\item E/A Stabilit�t q=0 und $T_\nu D_\nu > 0  \forall \nu$

\end{itemize}

\section{Betrachtung des Strecken-Frequenzgang}
\subsection{Definition:}
Der Frequenzgang (FG) eines LZI-Glieds ist dessen �F G(s) �ber der $j\omega$ Achse, d.h. \\
$G(s) \xrightarrow[s=j\omega]{} G(j\omega)$ \\
Zeitkonstantenform der �F: $\xrightarrow[]{s=j\omega} G(j\omega) = K \frac{(1+T_{z1} j\omega)...}{(j\omega)^2 (1+T_1j\omega)...}$ \\
Beispiel: $PT_1$-Glied:\\
$G(s) = \frac{K}{1+Ts} \xrightarrow{s=j\omega} G{(j\omega)} = \frac{K}{1+Tj\omega} = \frac{K}{1+T^2\omega^2} + j \frac{-KT\omega}{1+T^2\omega^2} 
      = \frac{K}{ \sqrt{\underbrace{1+T^2\omega^2}_{\left|G(j\omega)\right| }} } e^{j\underbrace{(-\arctan{(T\omega)})}_{\arg G(j\omega)}}$

\subsection{Bedeutung des FG f�r das System-Zeitverhalten}
\[u(t) = \hat{U} \sin{(\omega t)} \]
\[  y(t) = \left| G(j\omega) \right| \hat{U} \sin{(\omega_o t)} + \arg{G(j\omega_o)} + 
         \underbrace{\sum_{i=1}^n r_i(\omega_o) e^{p_i t}}_{\to 0 \text{ f�r } t \to \infty} \]

\begin{itemize}
 \item E/A stabil $\leftrightarrow \operatorname{Re}\lbrace p_i \rbrace < 0 \forall_i$
 \item  $\to$ im eingeschwungenen Zustand: $y(t) = \hat{y} \sin{(\omega_o + \varphi)} \text{ mit } \hat{y} = \left| G(j\omega_o) \right| \hat{u}, \varphi = \arg{G(j\omega)}$
 \item  D.h.: Der FG charakterisiert die �bertragung harmonischer Signale durch E/A stabile LZI-Systeme. Dabei beschreibt $\left| G(j\omega_o) \right|$ den 
	Amplitudengang und $\arg{G(j\omega)}$ den Phasengang. 
 \end{itemize}

\subsection{Graphische Darstellung des FG als Ortskurve in der komplexen $\Gamma$-Ebene}
\TODO{Grafik}
\begin{itemize}
 \item \underline{Ortskurve (OK)} zum Frequenzgang 
	$G(j\omega) \hat{=} $ Bild der $j\omega$-Achse der s-Ebene in der $\Gamma$-Ebene bei Anwendung der Abb. 
	$\Gamma = G(j\omega); -\infty \leq \omega \leq \infty$ 
 \item Beispiel: PT-1 Glied (FG siehe oben) OK-Verl�ufe h�ufiger �-Glieder siehe \bb{15}
 \item Da OK symetrisch zur reellen Achse, meist nur Darstellung der positiven H�lfte f�r $\omega \geq 0$
\end{itemize}

%\item Frequenzkennlinen weiterer h�ufiger �G: siehe \bb{15 + 16}. 

\subsection{Graphische Darstellung des FG als Frequenzkennlinen im Bode Diagramm}







