am Beispiel des Stellverhaltens mit der �F $G(s) = G_1 G_2 $

\section{Betrachtung der Strecken �F}
\subsection{Polynomform der �F}
\TODO{Formeln}

Da im allgemeinen \TODO{Formeln}, ist $\delta$ ein Ma� f�r die verz�gernde Wirkung eines Systems (siehe BB 13)

\subsection{Pol- Nullstellenform der �F}
Sind $n_1,...,n_m$ bzw. $p_i,...,p_n$ die (reellen oder konjugiert komplexen) Wurzeln von $Z(s)=0$ bzw. $N(s)=0$