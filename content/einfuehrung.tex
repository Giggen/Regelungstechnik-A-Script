Die Regelungstechnik (RT) besch�ftigt sich mit der selbstt�tigen gezielten Beeinflussung des
Verhaltens von dynamischen Systemen. 

\begin{center}
\begin{tikzpicture}[node distance=2.5cm,auto,>=latex']
	\node [system] 		(system) 						{System};
	\node 			(in) 	[left of=system,  node distance=4cm, coordinate] {system};
	\node [coordinate] 	(out) 	[right of=system, node distance=4cm]		{};

	\path[->,thick] (in) edge node {Eing�nge} (system);
	\draw[->,thick] (system) edge node {Ausg�nge} (out);
\end{tikzpicture}

\end{center}


Die Eing�nge werden in der RT aufgeteilt in von au�en vorgebbare \underline{Eingangsgr��en} und in durch die Umgebung festgelegte, st�rend wirkende \underline{St�rgr��en}. Von den Ausg�ngen werden nur diejenigen betrachtet, deren Verhalten unmittelbar interessiert. \underline{Ausgangsgr��en}.

\begin{center}
\begin{tikzpicture}[node distance=2.5cm,auto,>=latex']
	\node [system, pin={[arrowthick]above:St�rgr��en}] 	(system) 						{dynamisches System};
	\node 							(in) 	[left of=system,  node distance=6cm, coordinate]{system};
	\node [coordinate] 					(out) 	[right of=system, node distance=6cm]		{};

	\path[->,thick] (in) edge node {Eingangsgr��en} (system);
	\draw[->,thick] (system) edge node {Ausgangsgr��en} (out);
\end{tikzpicture}
\end{center}

\underline{Gezielte Beeinflussung} hei�t: Durch Vorgabe der Eingangsgr��enverl�ufe soll erreicht werden, dass die Ausgangsgr��en trotz St�reinwirkung ein gew�nschtes \underline{Sollverhalten} aufweisen.

\subsubsection*{Beispiel 1 (Raumtemperatur)}
\begin{center}
\begin{tikzpicture}[node distance=2.5cm,auto,>=latex']
	\node [system, pin={[arrowthick]above:Au�entemperatur, Sonne, Personen im Raum}]	
								(system) 						{Raum mit Heizk�rper und Ventil};
	\node 							(in) 	[left of=system,  node distance=6cm, coordinate]{system};
	\node [coordinate] 					(out) 	[right of=system, node distance=6cm]		{};

	\path[->,thick] (in) edge node {Heizk�rpereinstellung} (system);
	\draw[->,thick] (system) edge node {Raumtemperatur} (out);
\end{tikzpicture}
\end{center}

\subsubsection*{Beispiel 2 (Personenaufzug)}
\begin{center}
\begin{tikzpicture}[node distance=2.5cm,auto,>=latex']
	\node [system, pin={[arrowthick]above:Personen}] 	(system) 						{Aufzug mit el. Antrieb};
	\node 							(in) 	[left of=system,  node distance=6cm, coordinate]{system};
	\node [coordinate] 					(out) 	[right of=system, node distance=6cm]		{};

	\path[->,thick] (in) edge node {Strom} (system);
	\draw[->,thick] (system) edge node {Geschwindigkeit} (out);
\end{tikzpicture}
\end{center}

\subsubsection*{Beispiel 3 (Auto fahren)}
\begin{center}
\begin{tikzpicture}[node distance=2.5cm,auto,>=latex']
	\node [system, pin={[arrowthick]above:Witterung, Fahrbahn, Gewicht}] 	(system) 				{Fahrzeug};
	\node 							(in) 	[left of=system,  node distance=6cm, coordinate]{system};
	\node [coordinate] 					(out) 	[right of=system, node distance=6cm]		{};

	\path[->,thick] (in) edge node {Lenkradwinkel, Bremse, Gas} (system);
	\draw[->,thick] (system) edge node {Richtung, Geschwindigkeit} (out);
\end{tikzpicture}
\end{center}

\section*{Allgemeine Aufgabenstellung der RT}
Entwurf und Bereitstellung einer Einrichtung, die - hinzugef�gt zur Strecke - die Eingangsgr��en automatisch im gew�nschten Sinne generiert. \underline{Selbstt�tige gezielte Beeinflussung}.

\section*{Generelle Vorgehensweise zur L�sung}
\begin{enumerate}
	\item{mathematische Modellbildung der Strecke zur Abstraktion von deren physikalischen Auspr�gung und Erm�glichung der Anwendung universell einsetzbarer, systemtheoretisch fundierter Vorgehensweisen in Schritt 2. und 3.}
	\item{Analyse des Streckenverhaltens}
	\item{Entwurf der Steuer- und Regeleinrichtung}
	\item{Realisierung der Steuer- und Regeleinrichtung}
	\item{Inbetriebnahme und Erprobung des Gesamtsystems}
\end{enumerate}
